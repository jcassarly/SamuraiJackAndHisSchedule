%Copyright 2014 Jean-Philippe Eisenbarth
%This program is free software: you can 
%redistribute it and/or modify it under the terms of the GNU General Public 
%License as published by the Free Software Foundation, either version 3 of the 
%License, or (at your option) any later version.
%This program is distributed in the hope that it will be useful,but WITHOUT ANY 
%WARRANTY; without even the implied warranty of MERCHANTABILITY or FITNESS FOR A 
%PARTICULAR PURPOSE. See the GNU General Public License for more details.
%You should have received a copy of the GNU General Public License along with 
%this program.  If not, see <http://www.gnu.org/licenses/>.

%Based on the code of Yiannis Lazarides
%http://tex.stackexchange.com/questions/42602/software-requirements-specification-with-latex
%http://tex.stackexchange.com/users/963/yiannis-lazarides
%Also based on the template of Karl E. Wiegers
%http://www.se.rit.edu/~emad/teaching/slides/srs_template_sep14.pdf
%http://karlwiegers.com
\documentclass{scrreprt}
\usepackage{listings}
\usepackage{underscore}
\usepackage[bookmarks=true]{hyperref}
\usepackage[utf8]{inputenc}
\usepackage[english]{babel}
\usepackage{longtable}
\hypersetup{
    bookmarks=false,    % show bookmarks bar?
    pdftitle={Software Requirement Specification},    % title
    pdfauthor={Jean-Philippe Eisenbarth},                     % author
    pdfsubject={TeX and LaTeX},                        % subject of the document
    pdfkeywords={TeX, LaTeX, graphics, images}, % list of keywords
    colorlinks=true,       % false: boxed links; true: colored links
    linkcolor=blue,       % color of internal links
    citecolor=black,       % color of links to bibliography
    filecolor=black,        % color of file links
    urlcolor=purple,        % color of external links
    linktoc=page            % only page is linked
}%
\def\myversion{1.0 }
\date{}
%\title
\usepackage{hyperref}
\begin{document}

\begin{flushright}
    \rule{16cm}{5pt}\vskip1cm
    \begin{bfseries}
        \Huge{SOFTWARE REQUIREMENTS\\ SPECIFICATION}\\
        \vspace{1.9cm}
        for\\
        \vspace{1.9cm}
        GBUS Scheduling App\\
        \vspace{1.9cm}
        \LARGE{Version \myversion approved}\\
        \vspace{1.9cm}
        Prepared by Jack Doiron, Jared Cassarly, Shota Nemoto, Martin Peters\\
        \vspace{1.9cm}
        GBUS!LLC\\
        \vspace{1.9cm}
        \today\\
    \end{bfseries}
\end{flushright}

\tableofcontents


\chapter*{Revision History}

\begin{center}
    \begin{tabular}{|c|c|c|c|}
        \hline
	    Name & Date & Reason For Changes & Version\\
        \hline
	    21 & 22 & 23 & 24\\
        \hline
	    31 & 32 & 33 & 34\\
        \hline
    \end{tabular}
\end{center}

\chapter{Introduction}

\section{Purpose}
$<$Identify the product whose software requirements are specified in this 
document, including the revision or release number. Describe the scope of the 
product that is covered by this SRS, particularly if this SRS describes only 
part of the system or a single subsystem.$>$

\section{Document Conventions}
$<$Describe any standards or typographical conventions that were followed when 
writing this SRS, such as fonts or highlighting that have special significance.  
For example, state whether priorities  for higher-level requirements are assumed 
to be inherited by detailed requirements, or whether every requirement statement 
is to have its own priority.$>$

\section{Intended Audience and Reading Suggestions}
$<$Describe the different types of reader that the document is intended for, 
such as developers, project managers, marketing staff, users, testers, and 
documentation writers. Describe what the rest of this SRS contains and how it is 
organized. Suggest a sequence for reading the document, beginning with the 
overview sections and proceeding through the sections that are most pertinent to 
each reader type.$>$

\section{Project Scope}
$<$Provide a short description of the software being specified and its purpose, 
including relevant benefits, objectives, and goals. Relate the software to 
corporate goals or business strategies. If a separate vision and scope document 
is available, refer to it rather than duplicating its contents here.$>$

\section{References}
$<$List any other documents or Web addresses to which this SRS refers. These may 
include user interface style guides, contracts, standards, system requirements 
specifications, use case documents, or a vision and scope document. Provide 
enough information so that the reader could access a copy of each reference, 
including title, author, version number, date, and source or location.$>$


\chapter{Overall Description}

\section{Product Perspective}
$<$Describe the context and origin of the product being specified in this SRS.  
For example, state whether this product is a follow-on member of a product 
family, a replacement for certain existing systems, or a new, self-contained 
product. If the SRS defines a component of a larger system, relate the 
requirements of the larger system to the functionality of this software and 
identify interfaces between the two. A simple diagram that shows the major 
components of the overall system, subsystem interconnections, and external 
interfaces can be helpful.$>$

\section{Product Functions}
$<$Summarize the major functions the product must perform or must let the user 
perform. Details will be provided in Section 3, so only a high level summary 
(such as a bullet list) is needed here. Organize the functions to make them 
understandable to any reader of the SRS. A picture of the major groups of 
related requirements and how they relate, such as a top level data flow diagram 
or object class diagram, is often effective.$>$

\section{User Classes and Characteristics}
$<$Identify the various user classes that you anticipate will use this product.  
User classes may be differentiated based on frequency of use, subset of product 
functions used, technical expertise, security or privilege levels, educational 
level, or experience. Describe the pertinent characteristics of each user class.  
Certain requirements may pertain only to certain user classes. Distinguish the 
most important user classes for this product from those who are less important 
to satisfy.$>$

\section{Operating Environment}
$<$Describe the environment in which the software will operate, including the 
hardware platform, operating system and versions, and any other software 
components or applications with which it must peacefully coexist.$>$

\section{Design and Implementation Constraints}
$<$Describe any items or issues that will limit the options available to the 
developers. These might include: corporate or regulatory policies; hardware 
limitations (timing requirements, memory requirements); interfaces to other 
applications; specific technologies, tools, and databases to be used; parallel 
operations; language requirements; communications protocols; security 
considerations; design conventions or programming standards (for example, if the 
customer’s organization will be responsible for maintaining the delivered 
software).$>$

\section{User Documentation}
The user documentation includes: 
\begin{itemize}
    \item General help page\newline
    This page will provide links to other help pages specialized for particular
    features as well as answers to frequently asked questions.
    \item Event creation help page\newline
    This page will contain a step-by-step procedure of how to create events 
    and pictures of the relevant sections of the user interface.
    \item Event editing help page\newline
    This page will contain typical sequences for editing event attributes. It 
    includes information about the editing toolbar as well as editing events 
    from the event information screen.
    \item Auto-scheduling help page\newline
    This page will contain information about how the auto-scheduler identifies 
    time-slots to place its events as well as how each setting alters its behavior.
\end{itemize}

\section{Assumptions and Dependencies}

$<$List any assumed factors (as opposed to known facts) that could affect the 
requirements stated in the SRS. These could include third-party or commercial 
components that you plan to use, issues around the development or operating 
environment, or constraints. The project could be affected if these assumptions 
are incorrect, are not shared, or change. Also identify any dependencies the 
project has on external factors, such as software components that you intend to 
reuse from another project, unless they are already documented elsewhere (for 
example, in the vision and scope document or the project plan).$>$


\chapter{External Interface Requirements}

\section{User Interfaces}
$<$Describe the logical characteristics of each interface between the software 
product and the users. This may include sample screen images, any GUI standards 
or product family style guides that are to be followed, screen layout 
constraints, standard buttons and functions (e.g., help) that will appear on 
every screen, keyboard shortcuts, error message display standards, and so on.  
Define the software components for which a user interface is needed. Details of 
the user interface design should be documented in a separate user interface 
specification.$>$

\section{Hardware Interfaces}
$<$Describe the logical and physical characteristics of each interface between 
the software product and the hardware components of the system. This may include 
the supported device types, the nature of the data and control interactions 
between the software and the hardware, and communication protocols to be 
used.$>$

\section{Software Interfaces}
$<$Describe the connections between this product and other specific software 
components (name and version), including databases, operating systems, tools, 
libraries, and integrated commercial components. Identify the data items or 
messages coming into the system and going out and describe the purpose of each.  
Describe the services needed and the nature of communications. Refer to 
documents that describe detailed application programming interface protocols.  
Identify data that will be shared across software components. If the data 
sharing mechanism must be implemented in a specific way (for example, use of a 
global data area in a multitasking operating system), specify this as an 
implementation constraint.$>$

\section{Communications Interfaces}
$<$Describe the requirements associated with any communications functions 
required by this product, including e-mail, web browser, network server 
communications protocols, electronic forms, and so on. Define any pertinent 
message formatting. Identify any communication standards that will be used, such 
as FTP or HTTP. Specify any communication security or encryption issues, data 
transfer rates, and synchronization mechanisms.$>$


\chapter{System Features}
$<$This template illustrates organizing the functional requirements for the 
product by system features, the major services provided by the product. You may 
prefer to organize this section by use case, mode of operation, user class, 
object class, functional hierarchy, or combinations of these, whatever makes the 
most logical sense for your product.$>$

\section{GUI Calendar}
This provides a user with a graphical representation of their schedule based on a weekly view.

\subsection{Description and Priority}
This feature
$<$Provide a short description of the feature and indicate whether it is of 
High, Medium, or Low priority. You could also include specific priority 
component ratings, such as benefit, penalty, cost, and risk (each rated on a 
relative scale from a low of 1 to a high of 9).$>$

\subsection{Stimulus/Response Sequences}
$<$List the sequences of user actions and system responses that stimulate the 
behavior defined for this feature. These will correspond to the dialog elements 
associated with use cases.$>$

\subsection{Functional Requirements}
$<$Itemize the detailed functional requirements associated with this feature.  
These are the software capabilities that must be present in order for the user 
to carry out the services provided by the feature, or to execute the use case.  
Include how the product should respond to anticipated error conditions or 
invalid inputs. Requirements should be concise, complete, unambiguous, 
verifiable, and necessary. Use “TBD” as a placeholder to indicate when necessary 
information is not yet available.$>$

$<$Each requirement should be uniquely identified with a sequence number or a 
meaningful tag of some kind.$>$

REQ-1:	REQ-2:

\section{Inter-Device Synchronization}

\subsection{Description and Priority}
This feature allows the user to synchronize their schedule between devices. 
This will be done by allowing the user to register an account with which they
can log into the application. The user's schedule will be associated with that
account. All updates the user makes on their current device will be sent to the
server, so that their changes will be pulled from the server when the user logs 
in on a different device. Additionally, the user will be able to log out of 
their account on their current device. This feature will be \textbf{medium} priority.

\subsection{Stimulus/Response Sequences}
\begin{center}
    \begin{tabular}{ p{2cm} p{13cm} }
    Stimulus: & User opens application\\
    Response: & Application opens login dialog box for the user to enter their user-name
    and password.\\
    \\
    Stimulus: & User verifies login information\\
    Response: & If the information was correct, the user's schedule will be displayed.
    If not, the user will be informed it was incorrect and prompt for login information 
    again.\\
    \\
    Stimulus: & User attempts to register account\\
    Response: & Application opens up a new prompt where the user may create a new
    account and register it with the server.\\
    \\
    Stimulus: & User edits schedule locally\\
    Response: & Application sends the updates to the server and applies changes to 
    schedule on the server\\
    \\
    Stimulus: & User edits schedule on remote device\\
    Response: & Application receives the updates from the server and applies changes 
    schedule on local device.\\
    \\
    Stimulus: & User logs out of application\\
    Response: & Application closes the schedule and redisplays the login dialog box.\\
    \end{tabular}
\end{center}

\subsection{Functional Requirements}
\begin{center}
    \begin{longtable}{ | p{6cm} | p{9cm} | }
    \hline
    Sync.Login & The application must prompt the user for their login information
    upon startup.\\
    & \\
    Sync.Login.User-name & The login dialog box must allow the user to type their 
    user-name into a text field.\\
    & \\
    Sync.Login.Password & The login dialog box must allow the user to type their 
    password into a text field while hiding their entry.\\
    & \\
    Sync.Login.Verify & The login dialog box must allow the user to verify their
    entered login information. Application will query the server for the
    user-name and password pair. If it was correct they will see their schedule, 
    if not they will be prompted for their login information again.\\
    Sync.Login.Register & The login dialog box must open a new dialog box to allow the 
    user to create a new account and register it with the server\\
    \hline
    Sync.Update.Send & The application will send any updates made by the user to the 
    server at a scheduled interval. The updates will be applied to the schedule on the server.\\
    & \\
    Sync.Update.Receive & The application will receive any updates made to the schedule
    on the server and update the schedule on the local device if the schedules are different.\\
    \hline
    Sync.Logout & The application will allow the user to log out of their account. The
    application then displays the login dialog box. While closing the previous schedule\\
    \hline
    \end{longtable}
\end{center}

\section{Editing Toolbar}

\subsection{Description and Priority}
This feature allows the user to edit their schedule directly in the application's
graphical display. There will be a toolbar providing the user with editing tools
such as drag and drop, re-size, cut and paste, and add from existing. These editing
tools will be selected via the toolbar.

\subsection{Stimulus/Response Sequences}
\begin{center}
    \begin{tabular}{ p{2cm} p{13cm} }
    Stimulus: & User selects tool (clicking on tool's icon)\\
    Response: & Application registers subsequent clicks as using the selected tool and 
    adjusts events accordingly.\\
    \\
    Stimulus: & User uses drag and drop tool (click event then hold+drag)\\
    Response: & Application adjusts corresponding event's start and end times.\\
    \\
    Stimulus: & User uses re-size tool (click event edge then hold+drag top or bottom edge)\\
    Response: & Application adjusts corresponding event's start time if the top
    edge was moved, end time if the bottom edge was moved.\\
    \\
    Stimulus: & User uses copy and paste tool (click event then click on new time slot)\\
    Response: & Application selects event for "copying" and places a new copy of the event
    in the time slot where the user clicks. Subsequent clicks add additional copies of 
    the event. The original event is preserved.\\
    \\
    Stimulus: & User uses cut and paste tool (click event then click on new time slot)\\
    Response: & Application selects event for "cutting" and places a new copy of the event 
    in the time slot where the user clicks. Subsequent clicks add additional copies of 
    the event. The original event is removed.\\
    \\
    Stimulus: & User deselects tool (clicking on tool's icon again or selecting different tool)\\
    Response: & Application registers subsequent clicks as normal if tool's icon
    was clicked again. If the cut and paste tool was deselected, the "cut" event 
    is no longer pasted on clicks. If new tool was selected, registers subsequent clicks as 
    using the new tool. 
    \end{tabular}
\end{center}
The user will select a tool on the toolbar by clicking the tool's corresponding icon
then apply the tool to events on the schedule. The user can then utilize the tools 
by clicking on the events in their graphical user interface. The tools can be de-selected
by clicking on their icon in the toolbar.

\subsection{Functional Requirements}
\begin{center}
    \begin{longtable}{ | p{6cm} | p{9cm} | }
    \hline
    Toolbar.Select.DragDrop & The application registers the drag and drop tool
    as selected when the user clicks on the drag and drop icon in the toolbar.\\
    & \\
    Toolbar.Select.Re-size & The application registers the re-size tool as selected
    when the user clicks on the re-size icon in the toolbar.\\
    & \\
    Toolbar.Select.CutPaste & The application registers the cut and paste tool
    as selected when the user clicks on the cut and paste icon in the toolbar.\\
    \hline
    Toolbar.DragDrop & The application moves the start and end times to the new 
    times specified by the user when the drag and drop tool is selected\\
    & \\ 
    Toolbar.DragDrop.Drag & The application moves the event in the graphical
    interface following the user's cursor after the event was clicked on and
    not released.\\
    & \\
    Toolbar.DragDrop.Release & The application moves the start and end times to
    the new times specified by the time slot the user released the event in.\\
    \hline
    Toolbar.Re-size & The application adjusts the duration of the specified event.\\
    & \\
    Toolbar.Re-size.DragEdge & The application moves the top or bottom edge
    of the event in the graphical interface when the edge was clicked and not 
    released. The edge follows the cursor.\\
    & \\ 
    Toolbar.Re-size.Release & The application moves the start time to the new time
    specified by where the user released the top edge of the event. The end time
    will be moved if the bottom edge was released.\\
    \hline
    Toolbar.Copy & The application adds copies of an event at time slots specified
    by the user. The original event is preserved.\\
    & \\
    Toolbar.Copy.Select & The application selects the event for "copying".\\
    & \\
    Toolbar.Copy.Paste & The application creates a copy of the event selected
    for "copying" at the specified time slot.\\
    \hline
    Toolbar.Cut & The application adds copies of an event at time slots specified
    by the user. The original event is removed.\\
    & \\
    Toolbar.Cut.Select & The application selects the event for "cutting".\\
    & \\
    Toolbar.Cut.Paste & The application creates a copy of the event selected
    for "cutting" at the specified time slot.\\
    \hline
    \end{longtable}
\end{center}

\chapter{Other Nonfunctional Requirements}

\section{Performance Requirements}
$<$If there are performance requirements for the product under various 
circumstances, state them here and explain their rationale, to help the 
developers understand the intent and make suitable design choices. Specify the 
timing relationships for real time systems. Make such requirements as specific 
as possible. You may need to state performance requirements for individual 
functional requirements or features.$>$

\section{Safety Requirements}
$<$Specify those requirements that are concerned with possible loss, damage, or 
harm that could result from the use of the product. Define any safeguards or 
actions that must be taken, as well as actions that must be prevented. Refer to 
any external policies or regulations that state safety issues that affect the 
product’s design or use. Define any safety certifications that must be 
satisfied.$>$

\section{Security Requirements}
$<$Specify any requirements regarding security or privacy issues surrounding use 
of the product or protection of the data used or created by the product. Define 
any user identity authentication requirements. Refer to any external policies or 
regulations containing security issues that affect the product. Define any 
security or privacy certifications that must be satisfied.$>$

\section{Software Quality Attributes}
$<$Specify any additional quality characteristics for the product that will be 
important to either the customers or the developers. Some to consider are: 
adaptability, availability, correctness, flexibility, interoperability, 
maintainability, portability, reliability, reusability, robustness, testability, 
and usability. Write these to be specific, quantitative, and verifiable when 
possible. At the least, clarify the relative preferences for various attributes, 
such as ease of use over ease of learning.$>$

\section{Business Rules}
$<$List any operating principles about the product, such as which individuals or 
roles can perform which functions under specific circumstances. These are not 
functional requirements in themselves, but they may imply certain functional 
requirements to enforce the rules.$>$


\chapter{Other Requirements}
$<$Define any other requirements not covered elsewhere in the SRS. This might 
include database requirements, internationalization requirements, legal 
requirements, reuse objectives for the project, and so on. Add any new sections 
that are pertinent to the project.$>$

\section{Appendix A: Glossary}
%see https://en.wikibooks.org/wiki/LaTeX/Glossary
$<$Define all the terms necessary to properly interpret the SRS, including 
acronyms and abbreviations. You may wish to build a separate glossary that spans 
multiple projects or the entire organization, and just include terms specific to 
a single project in each SRS.$>$

\section{Appendix B: Analysis Models}
$<$Optionally, include any pertinent analysis models, such as data flow 
diagrams, class diagrams, state-transition diagrams, or entity-relationship 
diagrams.$>$

\section{Appendix C: To Be Determined List}
$<$Collect a numbered list of the TBD (to be determined) references that remain 
in the SRS so they can be tracked to closure.$>$

\end{document}