%Copyright 2014 Jean-Philippe Eisenbarth
%This program is free software: you can 
%redistribute it and/or modify it under the terms of the GNU General Public 
%License as published by the Free Software Foundation, either version 3 of the 
%License, or (at your option) any later version.
%This program is distributed in the hope that it will be useful,but WITHOUT ANY 
%WARRANTY; without even the implied warranty of MERCHANTABILITY or FITNESS FOR A 
%PARTICULAR PURPOSE. See the GNU General Public License for more details.
%You should have received a copy of the GNU General Public License along with 
%this program.  If not, see <http://www.gnu.org/licenses/>.

%Based on the code of Yiannis Lazarides
%http://tex.stackexchange.com/questions/42602/software-requirements-specification-with-latex
%http://tex.stackexchange.com/users/963/yiannis-lazarides
%Also based on the template of Karl E. Wiegers
%http://www.se.rit.edu/~emad/teaching/slides/srs_template_sep14.pdf
%http://karlwiegers.com
\documentclass{scrreprt}
\usepackage{listings}
\usepackage{underscore}
\usepackage[bookmarks=true]{hyperref}
\usepackage[utf8]{inputenc}
\usepackage[english]{babel}
\usepackage{longtable}
\hypersetup{
    bookmarks=false,    % show bookmarks bar?
    pdftitle={Software Requirement Specification},    % title
    pdfauthor={Jean-Philippe Eisenbarth},                     % author
    pdfsubject={TeX and LaTeX},                        % subject of the document
    pdfkeywords={TeX, LaTeX, graphics, images}, % list of keywords
    colorlinks=true,       % false: boxed links; true: colored links
    linkcolor=blue,       % color of internal links
    citecolor=black,       % color of links to bibliography
    filecolor=black,        % color of file links
    urlcolor=purple,        % color of external links
    linktoc=page            % only page is linked
}%
\def\myversion{1.0 }
\date{}
%\title
\usepackage{hyperref}
\begin{document}

\begin{flushright}
    \rule{16cm}{5pt}\vskip1cm
    \begin{bfseries}
        \Huge{SOFTWARE REQUIREMENTS\\ SPECIFICATION}\\
        \vspace{1.9cm}
        for\\
        \vspace{1.9cm}
        GBUS Scheduling App\\
        \vspace{1.9cm}
        \LARGE{Version \myversion approved}\\
        \vspace{1.9cm}
        Prepared by Jack Doiron, Jared Cassarly, Shota Nemoto, Martin Peters\\
        \vspace{1.9cm}
        GBUS!LLC\\
        \vspace{1.9cm}
        \today\\
    \end{bfseries}
\end{flushright}

\tableofcontents


\chapter*{Revision History}

\begin{center}
    \begin{tabular}{|c|c|c|c|}
        \hline
	    Name & Date & Reason For Changes & Version\\
        \hline
	    21 & 22 & 23 & 24\\
        \hline
	    31 & 32 & 33 & 34\\
        \hline
    \end{tabular}
\end{center}

\chapter{Introduction}

\section{Purpose}
$<$Identify the product whose software requirements are specified in this 
document, including the revision or release number. Describe the scope of the 
product that is covered by this SRS, particularly if this SRS describes only 
part of the system or a single subsystem.$>$

\section{Document Conventions}
$<$Describe any standards or typographical conventions that were followed when 
writing this SRS, such as fonts or highlighting that have special significance.  
For example, state whether priorities  for higher-level requirements are assumed 
to be inherited by detailed requirements, or whether every requirement statement 
is to have its own priority.$>$

\section{Intended Audience and Reading Suggestions}
$<$Describe the different types of reader that the document is intended for, 
such as developers, project managers, marketing staff, users, testers, and 
documentation writers. Describe what the rest of this SRS contains and how it is 
organized. Suggest a sequence for reading the document, beginning with the 
overview sections and proceeding through the sections that are most pertinent to 
each reader type.$>$

\section{Project Scope}
$<$Provide a short description of the software being specified and its purpose, 
including relevant benefits, objectives, and goals. Relate the software to 
corporate goals or business strategies. If a separate vision and scope document 
is available, refer to it rather than duplicating its contents here.$>$

\section{References}
$<$List any other documents or Web addresses to which this SRS refers. These may 
include user interface style guides, contracts, standards, system requirements 
specifications, use case documents, or a vision and scope document. Provide 
enough information so that the reader could access a copy of each reference, 
including title, author, version number, date, and source or location.$>$


\chapter{Overall Description}

\section{Product Perspective}
$<$Describe the context and origin of the product being specified in this SRS.  
For example, state whether this product is a follow-on member of a product 
family, a replacement for certain existing systems, or a new, self-contained 
product. If the SRS defines a component of a larger system, relate the 
requirements of the larger system to the functionality of this software and 
identify interfaces between the two. A simple diagram that shows the major 
components of the overall system, subsystem interconnections, and external 
interfaces can be helpful.$>$

\section{Product Functions}
$<$Summarize the major functions the product must perform or must let the user 
perform. Details will be provided in Section 3, so only a high level summary 
(such as a bullet list) is needed here. Organize the functions to make them 
understandable to any reader of the SRS. A picture of the major groups of 
related requirements and how they relate, such as a top level data flow diagram 
or object class diagram, is often effective.$>$

\section{User Classes and Characteristics}
$<$Identify the various user classes that you anticipate will use this product.  
User classes may be differentiated based on frequency of use, subset of product 
functions used, technical expertise, security or privilege levels, educational 
level, or experience. Describe the pertinent characteristics of each user class.  
Certain requirements may pertain only to certain user classes. Distinguish the 
most important user classes for this product from those who are less important 
to satisfy.$>$

\section{Operating Environment}
$<$Describe the environment in which the software will operate, including the 
hardware platform, operating system and versions, and any other software 
components or applications with which it must peacefully coexist.$>$

\section{Design and Implementation Constraints}
$<$Describe any items or issues that will limit the options available to the 
developers. These might include: corporate or regulatory policies; hardware 
limitations (timing requirements, memory requirements); interfaces to other 
applications; specific technologies, tools, and databases to be used; parallel 
operations; language requirements; communications protocols; security 
considerations; design conventions or programming standards (for example, if the 
customer’s organization will be responsible for maintaining the delivered 
software).$>$

\section{User Documentation}
$<$List the user documentation components (such as user manuals, on-line help, 
and tutorials) that will be delivered along with the software. Identify any 
known user documentation delivery formats or standards.$>$
\section{Assumptions and Dependencies}

$<$List any assumed factors (as opposed to known facts) that could affect the 
requirements stated in the SRS. These could include third-party or commercial 
components that you plan to use, issues around the development or operating 
environment, or constraints. The project could be affected if these assumptions 
are incorrect, are not shared, or change. Also identify any dependencies the 
project has on external factors, such as software components that you intend to 
reuse from another project, unless they are already documented elsewhere (for 
example, in the vision and scope document or the project plan).$>$


\chapter{External Interface Requirements}

\section{User Interfaces}
$<$Describe the logical characteristics of each interface between the software 
product and the users. This may include sample screen images, any GUI standards 
or product family style guides that are to be followed, screen layout 
constraints, standard buttons and functions (e.g., help) that will appear on 
every screen, keyboard shortcuts, error message display standards, and so on.  
Define the software components for which a user interface is needed. Details of 
the user interface design should be documented in a separate user interface 
specification.$>$

\section{Hardware Interfaces}
$<$Describe the logical and physical characteristics of each interface between 
the software product and the hardware components of the system. This may include 
the supported device types, the nature of the data and control interactions 
between the software and the hardware, and communication protocols to be 
used.$>$

\section{Software Interfaces}
$<$Describe the connections between this product and other specific software 
components (name and version), including databases, operating systems, tools, 
libraries, and integrated commercial components. Identify the data items or 
messages coming into the system and going out and describe the purpose of each.  
Describe the services needed and the nature of communications. Refer to 
documents that describe detailed application programming interface protocols.  
Identify data that will be shared across software components. If the data 
sharing mechanism must be implemented in a specific way (for example, use of a 
global data area in a multitasking operating system), specify this as an 
implementation constraint.$>$

\section{Communications Interfaces}
$<$Describe the requirements associated with any communications functions 
required by this product, including e-mail, web browser, network server 
communications protocols, electronic forms, and so on. Define any pertinent 
message formatting. Identify any communication standards that will be used, such 
as FTP or HTTP. Specify any communication security or encryption issues, data 
transfer rates, and synchronization mechanisms.$>$


\chapter{System Features}
$<$This template illustrates organizing the functional requirements for the 
product by system features, the major services provided by the product. You may 
prefer to organize this section by use case, mode of operation, user class, 
object class, functional hierarchy, or combinations of these, whatever makes the 
most logical sense for your product.$>$

\section{GUI Calendar}
This provides a user with a graphical representation of their schedule based on a weekly view.

\subsection{Description and Priority}
This feature
$<$Provide a short description of the feature and indicate whether it is of 
High, Medium, or Low priority. You could also include specific priority 
component ratings, such as benefit, penalty, cost, and risk (each rated on a 
relative scale from a low of 1 to a high of 9).$>$

\subsection{Stimulus/Response Sequences}
$<$List the sequences of user actions and system responses that stimulate the 
behavior defined for this feature. These will correspond to the dialog elements 
associated with use cases.$>$

\subsection{Functional Requirements}
$<$Itemize the detailed functional requirements associated with this feature.  
These are the software capabilities that must be present in order for the user 
to carry out the services provided by the feature, or to execute the use case.  
Include how the product should respond to anticipated error conditions or 
invalid inputs. Requirements should be concise, complete, unambiguous, 
verifiable, and necessary. Use “TBD” as a placeholder to indicate when necessary 
information is not yet available.$>$

$<$Each requirement should be uniquely identified with a sequence number or a 
meaningful tag of some kind.$>$

REQ-1:	REQ-2:

\section{Auto Scheduling}

\subsection{Description and Priority}
A user is able to create a Deadline Event or edit events such that they overlap.
When the user confirms these changes, the application will automatically reschedule
non-locked events, including any new Deadline Events being added such that events
do not conflict with each other and events occur at a valid location per the event.
If such a schedule is not found, the application will alert the user and will not
modify their schedule. When a new schedule has been created, the user may view a
graphical comparison with the old schedule and accept, modify, or reject it.
Priority=High.

\subsection{Stimulus/Response Sequences}
\begin{center}
\begin{tabular}{ p{2cm} p{13cm} }
Stimulus: & User adds Deadline events\\
Response: & Application searches for times that event blocks can be created for
the deadline event, moving non-locked events if needed.  All events must be valid
when rescheduling is finished.\\
\\
Stimulus: & User makes an edit which causes events to conflict.\\
Response: & Application leaves in place all events that were modified by the user
and attempts to reschedule conflicting events so that all events are valid.\\
\\
Stimulus: & A new schedule is generated by the Auto Scheduler\\
Response: & User is presented with a graphical comparison of their new and old
schedule and asked to either confirm, edit, or reject the schedule.\\
\\
Stimulus: & User accepts new schedule\\
Response: & The calendar is updated to reflect the new schedule.\\
\\
Stimulus: & User chooses to edit the new schedule\\
Response: & User is returned to edit mode with the new schedule and no changes
are made yet to the user calendar.\\
\\
Stimulus: & User rejects the new schedule\\
Response: & User is returned to their calendar and the new schedule changes
are thrown out.\\
\end{tabular}
\end{center}

\subsection{Functional Requirements}
\begin{center}
\begin{longtable}{ | p{6cm} | p{9cm} | }
\hline
Auto.Deadline & The auto scheduler must schedule time blocks for the user based on
the parameters given by a Deadline event\\
& \\
Auto.Deadline.Deadline & The auto scheduler must not schedule time blocks after the
User given deadline.\\
& \\
Auto.Deadline.TotalTime & The sum total time scheduled must be exactly equal to
the amount of time specified by the user.\\
& \\
Auto.Deadline.MinTime & The length of time that a time block is scheduled may not
be less than the minimum specified time by the user.\\
& \\
Auto.Deadline.MaxTime & The length of time that a time block is scheduled may not
be greater than the maximum specified time by the user.  In addition, any blocks
that are less than 1 hour apart count toward this time.\\
& \\
Auto.Deadline.Breaks & If specified by the user, time blocks must contain breaks
of a specified length such that there is no contiguous length of time greater
than specified by the user during which a block of the deadline event is scheduled
but no break is scheduled.  The minimum number of breaks must be used to achieve this.
Breaks count toward total block length, but do not contribute to the total time of
the deadline event.\\
\hline
Auto.Schedule & The auto scheduler must create a valid schedule.\\
& \\
Auto.Schedule.Deadline & All Deadline events in the schedule must conform to the
Auto.Deadline requirements.\\
& \\
Auto.Schedule.Conflicts & No non-location events may be scheduled at the same time\\
& \\
Auto.Schedule.Lock & Any events locked by the user may not be changed by the
auto scheduler.  All location events are treated as locked\\
& \\
Auto.Schedule.Location & Any events with a specified location must be scheduled
during corresponding location events\\
& \\
Auto.Schedule.MoveEvent & the auto scheduler may move events already scheduled by
the user as long as it conforms to the above requirements.\\
\hline
Auto.Diff & When the auto scheduler has created a valid schedule, the user is
presented with a graphical comparison of the new schedule with their current one
and asked if they want to accept it, edit it, or reject it.\\
& \\
Auto.Diff.Accept & If the user accepts the new schedule, the user's calendar is
updated to reflect these changes\\
& \\
Auto.Diff.Edit & If the user decides to edit the schedule, they are returned to
edit mode with the new schedule, but no changes are made to their calendar.\\
& \\
Auto.Diff.Reject & If the user rejects the schedule, they are returned to their
calendar view and the new schedule is thrown out with no changes being made to
their calendar.\\
\hline
\end{longtable}
\end{center}

\chapter{Other Nonfunctional Requirements}

\section{Performance Requirements}
\begin{center}
\begin{tabular}{ p{1.5cm} p{13cm} }
PE-1: & The user interface must respond to input within 100ms in order to maintain
the feeling of user control and fluency\\
PE-2: & The application must take less than 5 seconds to load completely and allow
for interaction across all platforms on a 5Mbps connection.
\end{tabular}
\end{center}

\section{Safety Requirements}
No safety Requirements have been identified

\section{Security Requirements}
\begin{center}
\begin{tabular}{ p{1.5cm} p{13cm} }
SE-1: & All communication between the server and client must be encrypted over https.\\
SE-2: & Plaintext passwords must not be stored on the server ever and should only be
stored in the database salted and hashed.\\
SE-3: & Users may only see their own information including but not limited to calendar
events and personal info such as name and email unless granted permission by the owner.
No other parties may have access to this information.
\end{tabular}
\end{center}

\section{Software Quality Attributes}
\begin{center}
\begin{tabular}{ p{1.5cm} p{13cm} }
QA-1: & 75\% of users must be able to create any type of event and use any of the
toolbar options after completing the first time tutorial.\\
QA-2: & The application must conform to all Android and iOS app standards for the
respective versions of the app.\\
QA-3: & If the app is closed while editing, the user should be returned to the editing
mode with the changes that they had made upon reopening the app on the same device.\\
QA-4: & 60\% of code for the app on any platform must be shared between all platforms.
\end{tabular}
\end{center}

\section{Other Requirements}
\begin{center}
\begin{tabular}{ p{1.5cm} p{13cm} }
OT-1: & Any non user generated text displayed to the user must be referenced from
a locales file which can contain alternatives for supported languages.\\
\end{tabular}
\end{center}

\chapter{Appendix A: Glossary}
%see https://en.wikibooks.org/wiki/LaTeX/Glossary
$<$Define all the terms necessary to properly interpret the SRS, including 
acronyms and abbreviations. You may wish to build a separate glossary that spans 
multiple projects or the entire organization, and just include terms specific to 
a single project in each SRS.$>$

\chapter{Appendix B: Analysis Models}
$<$Optionally, include any pertinent analysis models, such as data flow 
diagrams, class diagrams, state-transition diagrams, or entity-relationship 
diagrams.$>$

\chapter{Appendix C: To Be Determined List}
$<$Collect a numbered list of the TBD (to be determined) references that remain 
in the SRS so they can be tracked to closure.$>$

\end{document}